%% (Master) Thesis template
% Template version used: v1.4
%
% Largely adapted from Adrian Nievergelt's template for the ADPS
% (lecture notes) project.


%% We use the memoir class because it offers a many easy to use features.
\documentclass[11pt,a4paper,titlepage]{memoir}

%% Packages
%% ========

%% LaTeX Font encoding -- DO NOT CHANGE
\usepackage[OT1]{fontenc}

%% Babel provides support for languages.  'english' uses British
%% English hyphenation and text snippets like "Figure" and
%% "Theorem". Use the option 'ngerman' if your document is in German.
%% Use 'american' for American English.  Note that if you change this,
%% the next LaTeX run may show spurious errors.  Simply run it again.
%% If they persist, remove the .aux file and try again.
\usepackage[english]{babel}

%% Input encoding 'utf8'. In some cases you might need 'utf8x' for
%% extra symbols. Not all editors, especially on Windows, are UTF-8
%% capable, so you may want to use 'latin1' instead.
\usepackage[utf8]{inputenc}

%% This changes default fonts for both text and math mode to use Herman Zapfs
%% excellent Palatino font.  Do not change this.
\usepackage[sc]{mathpazo}

%% The AMS-LaTeX extensions for mathematical typesetting.  Do not
%% remove.
\usepackage{amsmath,amssymb,amsfonts,mathrsfs}

%% NTheorem is a reimplementation of the AMS Theorem package. This
%% will allow us to typeset theorems like examples, proofs and
%% similar.  Do not remove.
%% NOTE: Must be loaded AFTER amsmath, or the \qed placement will
%% break
\usepackage[amsmath,thmmarks]{ntheorem}

%% LaTeX' own graphics handling
\usepackage{graphicx}

%% Be able to draw svg's.
\usepackage{svg}

%% We unfortunately need this for the Rules chapter.  Remove it
%% afterwards; or at least NEVER use its underlining features.
\usepackage{soul}

%% This allows you to add .pdf files. It is used to add the
%% declaration of originality.
\usepackage{pdfpages}

%% Some more packages that you may want to use.  Have a look at the
%% file, and consult the package docs for each.
\input{extrapackages}

%% Our layout configuration.  DO NOT CHANGE.
\input{layoutsetup}

%% Theorem environments.  You will have to adapt this for a German
%% thesis.
\input{theoremsetup}

%% Helpful macros.
\input{macrosetup}

\graphicspath{ {./images/} }


%% Make document internal hyperlinks wherever possible. (TOC, references)
%% This MUST be loaded after varioref, which is loaded in 'extrapackages'
%% above.  We just load it last to be safe.
\usepackage[linkcolor=black,colorlinks=true,citecolor=black,filecolor=black]{hyperref}


%% Document information
%% ====================

\title{Multi-model disentanglement of object representation}
\author{Eric Stavarache}
\thesistype{Master Thesis}
\advisors{Advisors: Frederik Benzing, Asier Mujika}
\department{Department of Computer Science}
\date{July 19, 2019}

\begin{document}

\frontmatter

%% Title page is autogenerated from document information above.  DO
%% NOT CHANGE.
\begin{titlingpage}
  \calccentering{\unitlength}
  \begin{adjustwidth*}{\unitlength-24pt}{-\unitlength-24pt}
    \maketitle
  \end{adjustwidth*}
\end{titlingpage}

%% The abstract of your thesis.  Edit the file as needed.
\input{abstract}

%% TOC with the proper setup, do not change.
\cleartorecto
\tableofcontents
\mainmatter

\chapter{Introduction}

Humans reason about the surrounding world by decomposing it into orthogonal components.
The idea of an object is decoupled from the qualities of the object: it is very
easy for us to imagine a pink elephant, even though we have never observed such
an animal, and these two words suffice for us to imagine a visual scene.
The words are a latent representation of the scene.

Variational Autoencoders (VAE) \cite{bib:vae_paper} are a powerful framework for
learning latent representations. Much work has already been done on disentangled
representations, where we require that any two latent variables are uncorrelated.

Our work focuses on model-based disentanglement of objects. In particular, we
would like to have one individual VAE which is trained to recognise and
represent a single type of object: in a normal setting, we could have one VAE
which represents chairs, another VAE which represents tables, and so on.
We will call our approach the \textbf{SuperVAE}.

Most similar in spirit is the \cite{bib:monet}, where a single VAE model
iteratively recognises objects from a scene by using attention masks.
The main difference is that the MONet uses a single VAE for all objects, whereas
we use multiple VAE's in parallel.
Another similar idea is contained in \cite{bib:iodine}, where they have a single
VAE which tries to model the scene, and where they iteratively refine the latent
parameters.

\chapter{Method}

\section{SuperVAE}


The SuperVAE network tries to reconstructor a scene by feeding the image to some
VAE's connected in parallel, whereby every VAE models a separate part of the scene.


Let us say that there are $K$ VAE's, where each of them tries to model a
different object.
Then the image $X$ will be fed independently to
each of them.

Each VAE will then model a distribution over the latent variables,
$q_{\theta_k}(\textbf{z}_k | \textbf{X})$.

By sampling from these distribution, they will generate two outputs of
the same size as the image: $\hat{X}_k$, which is the $k$'th model
reconstruction of the image, and $\hat{m}_{k}$, which is a confidence mask.

This confidence mask represents how sure a VAE is about its output for a given
pixel. Higher confidence means that the pixel is part of an object which is of
the type the VAE is modelling.

These confidence masks are produced by first having the models generate some
``raw'' confidence values, and then taking a pixel-wise softmax across all of
the $k$ models.

Thus, the masks are a probability distribution, where for each pixel we have
what is the probability that it belongs to an object modelled by a VAE:

$\sum_{i = 1}^{k} \hat{m}_{k} = \textbf{1}$.

\begin{figure}
  \caption{A diagram of how the model operates.}
  \includegraphics[scale=0.7]{model.png}
\end{figure}

Our loss function is

\[
  \mathcal{L}(X; \theta _1, ..., \theta_k) =
  \sum_{i = 1}^{k}
  \norm {
    (\hat{X}_{i} - X)
    \odot \hat{m}_{i}
  }^2_F
  +
  \beta \sum_{i = 1}^{k} D_{KL} [ q_{\theta_k} (\textbf{z}_k | \textbf{X}) ||
  \mathcal{N}(0, \mathbf{I}) ]
  +
  \gamma \cdot - \log (
  \frac{1} {\log{K}}
  \cdot - \hat{m} \log {\hat{m}}
  )
\]

The first term of the loss is a $L_2$ loss weighted by each model's confidence.
The intuition behind this is that the more responsability a VAE takes for
drawing certain pixel, the more it should be penalized for making mistakes.

The second term is the standard KL loss term, weighted by a $\beta$
hyperparameter, as first introduced in \cite{bib:betavae}.

The last term is our original contribution, which is a cross-entropy loss.
The idea is that the VAE's should try to be unassuming, and they should incur a
cost if they decide to take on a lot of responsability for representing a pixel.
We have the $-\hat{m} \log {\hat{m}}$ term, which is the normal cross-entropy,
where we interpret the masks as component-wise distributions over VAE's.
We divide it by the maximal value it can take ($\log{K}$, where $K$ is the
number of VAE's) in order to obtain a value between $0$ and $1$.
We want to penalize situations where one VAE takes over everything, and in these
situations the cross-entropy would be close to $0$. As such, we take the
negative log of the normalized cross-entropy.
This whole term is weighted by another hyperparameter, $\gamma$.


We compare this with the approach of \cite{bib:monet}, where they have a single
VAE which tries to model all objects, whereas we have a single VAE for each
object type.
Furthermore, in their model, the VAE is instructed what to model by the U-Net
attention mask, whereas in our approach each VAE decides ``independently'' what
to learn. In our approach, the only communication happening between the VAE's is
via the softmax operation of the confidence masks.


\chapter{Results}

For our experiments, we have taken a supervised approach to training the models.

\section{MNIST}

In order to test our idea, we began with the MNIST dataset.
To make the task more challenging, and to allow decomposition via objects, we
construct our training data by putting two digits side-by-side.

\subsection{2-MNIST}

At the start, we set up only 2 VAE's in parallel, where we want that VAE-0 learns to
recognise and model the digit 0, and VAE-1 models digit 1.

In order to accomplish this, we split the training into two stages:

\begin{enumerate}
  \item In stage 0, VAE-0 trains on reconstructing images containing
    only the digit $0$. The other model is active (so they are contributing
    to the confidence masks), but its weights are frozen, and only
    VAE-0 is learning.
  \item In stage 1, VAE-1 trains on images containing digits $0$ and $1$ (so of
    the two slots, each digit is sampled independently). In this time, VAE-0 is
    frozen, \textit{however} it is still contributing to the confidence masks.
    Because of this, places where the digit $0$ appears are already assigned
    high confidence values by VAE-0, and so VAE-1 will not be motivated to learn
    them. On the other hand, places where digit 1 appears will not be recognised
    by VAE-0, and so VAE-1 will learn them.
\end{enumerate}

\begin{figure}
  \caption{Results of training on 2-MNIST.
    Each picture has 6 rows: first row is input image $X$, second row is
    confidence mask of VAE-0 $\hat{m}_{0}$, third row is reconstruction of VAE-0
    $\hat{X}_{0}$, fourth row is $\hat{m}_{1}$, fifth row is $\hat{X}_{1}$, and
    last row is weighted reconstruction.
  }
  \centering
  \includegraphics[scale=0.3]{init_progress.png}
\end{figure}


\appendix

\input{appendix}

\backmatter

\bibliographystyle{plain}
\bibliography{refs}

% \includepdf[pages={-}]{declaration-originality.pdf}

\end{document}
